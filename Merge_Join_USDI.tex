% Options for packages loaded elsewhere
\PassOptionsToPackage{unicode}{hyperref}
\PassOptionsToPackage{hyphens}{url}
\PassOptionsToPackage{dvipsnames,svgnames,x11names}{xcolor}
%
\documentclass[
]{article}
\usepackage{amsmath,amssymb}
\usepackage{iftex}
\ifPDFTeX
  \usepackage[T1]{fontenc}
  \usepackage[utf8]{inputenc}
  \usepackage{textcomp} % provide euro and other symbols
\else % if luatex or xetex
  \usepackage{unicode-math} % this also loads fontspec
  \defaultfontfeatures{Scale=MatchLowercase}
  \defaultfontfeatures[\rmfamily]{Ligatures=TeX,Scale=1}
\fi
\usepackage{lmodern}
\ifPDFTeX\else
  % xetex/luatex font selection
\fi
% Use upquote if available, for straight quotes in verbatim environments
\IfFileExists{upquote.sty}{\usepackage{upquote}}{}
\IfFileExists{microtype.sty}{% use microtype if available
  \usepackage[]{microtype}
  \UseMicrotypeSet[protrusion]{basicmath} % disable protrusion for tt fonts
}{}
\makeatletter
\@ifundefined{KOMAClassName}{% if non-KOMA class
  \IfFileExists{parskip.sty}{%
    \usepackage{parskip}
  }{% else
    \setlength{\parindent}{0pt}
    \setlength{\parskip}{6pt plus 2pt minus 1pt}}
}{% if KOMA class
  \KOMAoptions{parskip=half}}
\makeatother
\usepackage{xcolor}
\usepackage[margin=1in]{geometry}
\usepackage{color}
\usepackage{fancyvrb}
\newcommand{\VerbBar}{|}
\newcommand{\VERB}{\Verb[commandchars=\\\{\}]}
\DefineVerbatimEnvironment{Highlighting}{Verbatim}{commandchars=\\\{\}}
% Add ',fontsize=\small' for more characters per line
\usepackage{framed}
\definecolor{shadecolor}{RGB}{248,248,248}
\newenvironment{Shaded}{\begin{snugshade}}{\end{snugshade}}
\newcommand{\AlertTok}[1]{\textcolor[rgb]{0.94,0.16,0.16}{#1}}
\newcommand{\AnnotationTok}[1]{\textcolor[rgb]{0.56,0.35,0.01}{\textbf{\textit{#1}}}}
\newcommand{\AttributeTok}[1]{\textcolor[rgb]{0.13,0.29,0.53}{#1}}
\newcommand{\BaseNTok}[1]{\textcolor[rgb]{0.00,0.00,0.81}{#1}}
\newcommand{\BuiltInTok}[1]{#1}
\newcommand{\CharTok}[1]{\textcolor[rgb]{0.31,0.60,0.02}{#1}}
\newcommand{\CommentTok}[1]{\textcolor[rgb]{0.56,0.35,0.01}{\textit{#1}}}
\newcommand{\CommentVarTok}[1]{\textcolor[rgb]{0.56,0.35,0.01}{\textbf{\textit{#1}}}}
\newcommand{\ConstantTok}[1]{\textcolor[rgb]{0.56,0.35,0.01}{#1}}
\newcommand{\ControlFlowTok}[1]{\textcolor[rgb]{0.13,0.29,0.53}{\textbf{#1}}}
\newcommand{\DataTypeTok}[1]{\textcolor[rgb]{0.13,0.29,0.53}{#1}}
\newcommand{\DecValTok}[1]{\textcolor[rgb]{0.00,0.00,0.81}{#1}}
\newcommand{\DocumentationTok}[1]{\textcolor[rgb]{0.56,0.35,0.01}{\textbf{\textit{#1}}}}
\newcommand{\ErrorTok}[1]{\textcolor[rgb]{0.64,0.00,0.00}{\textbf{#1}}}
\newcommand{\ExtensionTok}[1]{#1}
\newcommand{\FloatTok}[1]{\textcolor[rgb]{0.00,0.00,0.81}{#1}}
\newcommand{\FunctionTok}[1]{\textcolor[rgb]{0.13,0.29,0.53}{\textbf{#1}}}
\newcommand{\ImportTok}[1]{#1}
\newcommand{\InformationTok}[1]{\textcolor[rgb]{0.56,0.35,0.01}{\textbf{\textit{#1}}}}
\newcommand{\KeywordTok}[1]{\textcolor[rgb]{0.13,0.29,0.53}{\textbf{#1}}}
\newcommand{\NormalTok}[1]{#1}
\newcommand{\OperatorTok}[1]{\textcolor[rgb]{0.81,0.36,0.00}{\textbf{#1}}}
\newcommand{\OtherTok}[1]{\textcolor[rgb]{0.56,0.35,0.01}{#1}}
\newcommand{\PreprocessorTok}[1]{\textcolor[rgb]{0.56,0.35,0.01}{\textit{#1}}}
\newcommand{\RegionMarkerTok}[1]{#1}
\newcommand{\SpecialCharTok}[1]{\textcolor[rgb]{0.81,0.36,0.00}{\textbf{#1}}}
\newcommand{\SpecialStringTok}[1]{\textcolor[rgb]{0.31,0.60,0.02}{#1}}
\newcommand{\StringTok}[1]{\textcolor[rgb]{0.31,0.60,0.02}{#1}}
\newcommand{\VariableTok}[1]{\textcolor[rgb]{0.00,0.00,0.00}{#1}}
\newcommand{\VerbatimStringTok}[1]{\textcolor[rgb]{0.31,0.60,0.02}{#1}}
\newcommand{\WarningTok}[1]{\textcolor[rgb]{0.56,0.35,0.01}{\textbf{\textit{#1}}}}
\usepackage{longtable,booktabs,array}
\usepackage{calc} % for calculating minipage widths
% Correct order of tables after \paragraph or \subparagraph
\usepackage{etoolbox}
\makeatletter
\patchcmd\longtable{\par}{\if@noskipsec\mbox{}\fi\par}{}{}
\makeatother
% Allow footnotes in longtable head/foot
\IfFileExists{footnotehyper.sty}{\usepackage{footnotehyper}}{\usepackage{footnote}}
\makesavenoteenv{longtable}
\usepackage{graphicx}
\makeatletter
\def\maxwidth{\ifdim\Gin@nat@width>\linewidth\linewidth\else\Gin@nat@width\fi}
\def\maxheight{\ifdim\Gin@nat@height>\textheight\textheight\else\Gin@nat@height\fi}
\makeatother
% Scale images if necessary, so that they will not overflow the page
% margins by default, and it is still possible to overwrite the defaults
% using explicit options in \includegraphics[width, height, ...]{}
\setkeys{Gin}{width=\maxwidth,height=\maxheight,keepaspectratio}
% Set default figure placement to htbp
\makeatletter
\def\fps@figure{htbp}
\makeatother
\setlength{\emergencystretch}{3em} % prevent overfull lines
\providecommand{\tightlist}{%
  \setlength{\itemsep}{0pt}\setlength{\parskip}{0pt}}
\setcounter{secnumdepth}{5}
\ifLuaTeX
  \usepackage{selnolig}  % disable illegal ligatures
\fi
\IfFileExists{bookmark.sty}{\usepackage{bookmark}}{\usepackage{hyperref}}
\IfFileExists{xurl.sty}{\usepackage{xurl}}{} % add URL line breaks if available
\urlstyle{same}
\hypersetup{
  pdftitle={(USDI): Merge, Join and Preparing Data for Plots},
  pdfauthor={Abiola Oyebanjo},
  colorlinks=true,
  linkcolor={Maroon},
  filecolor={Maroon},
  citecolor={Blue},
  urlcolor={blue},
  pdfcreator={LaTeX via pandoc}}

\title{(USDI): Merge, Join and Preparing Data for Plots}
\author{Abiola Oyebanjo}
\date{July 4-6 , 2024}

\begin{document}
\maketitle

{
\hypersetup{linkcolor=}
\setcounter{tocdepth}{2}
\tableofcontents
}
\begin{center}\rule{0.5\linewidth}{0.5pt}\end{center}

\hypertarget{import-libraries-dataframes}{%
\section{Import Libraries
Dataframes}\label{import-libraries-dataframes}}

\begin{Shaded}
\begin{Highlighting}[]
\FunctionTok{library}\NormalTok{(dplyr)}
\FunctionTok{library}\NormalTok{(tidyverse)}
\FunctionTok{library}\NormalTok{(reshape2) }\CommentTok{\# for melt}
\end{Highlighting}
\end{Shaded}

\hypertarget{set-working-directory}{%
\section{Set Working Directory}\label{set-working-directory}}

\hypertarget{joining-dataframes}{%
\section{Joining Dataframes}\label{joining-dataframes}}

\hypertarget{joining-dataframes-1}{%
\subsection{Joining dataframes}\label{joining-dataframes-1}}

Joining dataframes is a common operation when working with relational
data. It allows us to combine information from different sources based
on common variables.

\hypertarget{creating-sample-dataframes}{%
\subsubsection{Creating Sample
Dataframes}\label{creating-sample-dataframes}}

Let's start by creating two simple dataframes:

\begin{Shaded}
\begin{Highlighting}[]
\NormalTok{df1 }\OtherTok{\textless{}{-}} \FunctionTok{data.frame}\NormalTok{(}
  \AttributeTok{ID =} \DecValTok{1}\SpecialCharTok{:}\DecValTok{5}\NormalTok{,}
  \AttributeTok{Name =} \FunctionTok{c}\NormalTok{(}\StringTok{"Alice"}\NormalTok{, }\StringTok{"Bob"}\NormalTok{, }\StringTok{"Charlie"}\NormalTok{, }\StringTok{"David"}\NormalTok{, }\StringTok{"Eve"}\NormalTok{)}
\NormalTok{)}

\NormalTok{df2 }\OtherTok{\textless{}{-}} \FunctionTok{data.frame}\NormalTok{(}
  \AttributeTok{ID =} \DecValTok{2}\SpecialCharTok{:}\DecValTok{6}\NormalTok{,}
  \AttributeTok{Score =} \FunctionTok{c}\NormalTok{(}\DecValTok{85}\NormalTok{, }\DecValTok{92}\NormalTok{, }\DecValTok{78}\NormalTok{, }\DecValTok{95}\NormalTok{, }\DecValTok{88}\NormalTok{)}
\NormalTok{)}

\FunctionTok{print}\NormalTok{(df1)}
\end{Highlighting}
\end{Shaded}

\begin{verbatim}
##   ID    Name
## 1  1   Alice
## 2  2     Bob
## 3  3 Charlie
## 4  4   David
## 5  5     Eve
\end{verbatim}

\begin{Shaded}
\begin{Highlighting}[]
\FunctionTok{print}\NormalTok{(df2)}
\end{Highlighting}
\end{Shaded}

\begin{verbatim}
##   ID Score
## 1  2    85
## 2  3    92
## 3  4    78
## 4  5    95
## 5  6    88
\end{verbatim}

Here, we've created two dataframes:

\texttt{df1} contains student IDs and names \texttt{df2} contains
student IDs and scores

Notice that the IDs don't perfectly align between the two dataframes.
This is intentional to demonstrate different join behaviors.

\hypertarget{using-merge-from-base-r}{%
\subsection{Using merge() from Base R}\label{using-merge-from-base-r}}

The \texttt{merge()} function in base R is versatile and can perform
various types of joins.

\hypertarget{left-join}{%
\subsection{Left Join}\label{left-join}}

\begin{Shaded}
\begin{Highlighting}[]
\NormalTok{left\_merge }\OtherTok{\textless{}{-}} \FunctionTok{merge}\NormalTok{(df1, df2, }\AttributeTok{by =} \StringTok{"ID"}\NormalTok{, }\AttributeTok{all.x =} \ConstantTok{TRUE}\NormalTok{)}
\FunctionTok{print}\NormalTok{(left\_merge)}
\end{Highlighting}
\end{Shaded}

\begin{verbatim}
##   ID    Name Score
## 1  1   Alice    NA
## 2  2     Bob    85
## 3  3 Charlie    92
## 4  4   David    78
## 5  5     Eve    95
\end{verbatim}

Explanation:

\texttt{by\ =\ "ID"} specifies that we're joining on the ``ID'' column

\texttt{all.x\ =\ TRUE} means we keep all rows from the left dataframe
(df1), even if there's no match in df2

This results in NA values for scores where there's no match in df2

\hypertarget{right-join}{%
\subsection{Right Join}\label{right-join}}

\begin{Shaded}
\begin{Highlighting}[]
\NormalTok{right\_merge }\OtherTok{\textless{}{-}} \FunctionTok{merge}\NormalTok{(df1, df2, }\AttributeTok{by =} \StringTok{"ID"}\NormalTok{, }\AttributeTok{all.y =} \ConstantTok{TRUE}\NormalTok{)}
\FunctionTok{print}\NormalTok{(right\_merge)}
\end{Highlighting}
\end{Shaded}

\begin{verbatim}
##   ID    Name Score
## 1  2     Bob    85
## 2  3 Charlie    92
## 3  4   David    78
## 4  5     Eve    95
## 5  6    <NA>    88
\end{verbatim}

Explanation:

\texttt{all.y} = TRUE means we keep all rows from the right dataframe
(df2), even if there's no match in df1 This results in NA values for
names where there's no match in df1

\hypertarget{inner-join}{%
\subsection{Inner Join}\label{inner-join}}

\begin{Shaded}
\begin{Highlighting}[]
\NormalTok{inner\_merge }\OtherTok{\textless{}{-}} \FunctionTok{merge}\NormalTok{(df1, df2, }\AttributeTok{by =} \StringTok{"ID"}\NormalTok{)}
\FunctionTok{print}\NormalTok{(inner\_merge)}
\end{Highlighting}
\end{Shaded}

\begin{verbatim}
##   ID    Name Score
## 1  2     Bob    85
## 2  3 Charlie    92
## 3  4   David    78
## 4  5     Eve    95
\end{verbatim}

Explanation:

Without \texttt{all.x} or \texttt{all.y}, merge() performs an inner join
by default This keeps only the rows where there's a match in both
dataframes

\hypertarget{full-outer-join}{%
\subsection{Full Outer Join}\label{full-outer-join}}

\begin{Shaded}
\begin{Highlighting}[]
\NormalTok{full\_merge }\OtherTok{\textless{}{-}} \FunctionTok{merge}\NormalTok{(df1, df2, }\AttributeTok{by =} \StringTok{"ID"}\NormalTok{, }\AttributeTok{all =} \ConstantTok{TRUE}\NormalTok{)}
\FunctionTok{print}\NormalTok{(full\_merge)}
\end{Highlighting}
\end{Shaded}

\begin{verbatim}
##   ID    Name Score
## 1  1   Alice    NA
## 2  2     Bob    85
## 3  3 Charlie    92
## 4  4   David    78
## 5  5     Eve    95
## 6  6    <NA>    88
\end{verbatim}

Explanation:

\texttt{all\ =\ TRUE} keeps all rows from both dataframes, filling in NA
where there's no match This is useful when you want to see all data from
both sources, regardless of matches

\hypertarget{using-dplyr-for-joins}{%
\section{Using dplyr for Joins}\label{using-dplyr-for-joins}}

The dplyr package provides more intuitive join functions that are often
preferred in modern R programming.

\hypertarget{left-join-1}{%
\subsection{Left Join}\label{left-join-1}}

\begin{Shaded}
\begin{Highlighting}[]
\NormalTok{left\_join\_dplyr }\OtherTok{\textless{}{-}} \FunctionTok{left\_join}\NormalTok{(df1, df2, }\AttributeTok{by =} \StringTok{"ID"}\NormalTok{)}
\FunctionTok{print}\NormalTok{(left\_join\_dplyr)}
\end{Highlighting}
\end{Shaded}

\begin{verbatim}
##   ID    Name Score
## 1  1   Alice    NA
## 2  2     Bob    85
## 3  3 Charlie    92
## 4  4   David    78
## 5  5     Eve    95
\end{verbatim}

Explanation:

\texttt{left\_join()} keeps all rows from df1 and adds matching data
from df2 It's equivalent to the left merge we did earlier, but with a
more readable syntax

\hypertarget{right-join-1}{%
\subsection{Right Join}\label{right-join-1}}

\begin{Shaded}
\begin{Highlighting}[]
\NormalTok{left\_join\_dplyr }\OtherTok{\textless{}{-}} \FunctionTok{left\_join}\NormalTok{(df1, df2, }\AttributeTok{by =} \StringTok{"ID"}\NormalTok{)}
\FunctionTok{print}\NormalTok{(left\_join\_dplyr)}
\end{Highlighting}
\end{Shaded}

\begin{verbatim}
##   ID    Name Score
## 1  1   Alice    NA
## 2  2     Bob    85
## 3  3 Charlie    92
## 4  4   David    78
## 5  5     Eve    95
\end{verbatim}

Explanation:

\texttt{right\_join()} keeps all rows from df2 and adds matching data
from df1 It's equivalent to the right merge we did earlier

\hypertarget{inner-join-1}{%
\subsection{Inner Join}\label{inner-join-1}}

\begin{Shaded}
\begin{Highlighting}[]
\NormalTok{inner\_join\_dplyr }\OtherTok{\textless{}{-}} \FunctionTok{inner\_join}\NormalTok{(df1, df2, }\AttributeTok{by =} \StringTok{"ID"}\NormalTok{)}
\FunctionTok{print}\NormalTok{(inner\_join\_dplyr)}
\end{Highlighting}
\end{Shaded}

\begin{verbatim}
##   ID    Name Score
## 1  2     Bob    85
## 2  3 Charlie    92
## 3  4   David    78
## 4  5     Eve    95
\end{verbatim}

Explanation:

\texttt{inner\_join()} keeps only rows with matches in both dataframes
It's equivalent to the inner merge we did earlier

\hypertarget{full-join}{%
\subsection{Full Join}\label{full-join}}

\begin{Shaded}
\begin{Highlighting}[]
\NormalTok{full\_join\_dplyr }\OtherTok{\textless{}{-}} \FunctionTok{full\_join}\NormalTok{(df1, df2, }\AttributeTok{by =} \StringTok{"ID"}\NormalTok{)}
\FunctionTok{print}\NormalTok{(full\_join\_dplyr)}
\end{Highlighting}
\end{Shaded}

\begin{verbatim}
##   ID    Name Score
## 1  1   Alice    NA
## 2  2     Bob    85
## 3  3 Charlie    92
## 4  4   David    78
## 5  5     Eve    95
## 6  6    <NA>    88
\end{verbatim}

Explanation:

\texttt{full\_join()} keeps all rows from both dataframes, filling in NA
where there's no match It's equivalent to the full outer merge we did
earlier

\hypertarget{reshaping-data-for-visualization-and-panel-data-analysis}{%
\section{Reshaping Data (for Visualization and Panel Data
Analysis)}\label{reshaping-data-for-visualization-and-panel-data-analysis}}

Reshaping data involves changing the structure of a dataset without
changing the information it contains. The two main forms are ``wide''
and ``long'' formats. This is very helpful for visualization purposes
and Panel (Longitudinal) data analysis. Longitudinal data involves
repeated observations of the same variables over time for the same
subjects. This type of data allows for the analysis of changes over time
and the study of temporal dynamics within the data.

\hypertarget{wide-to-long-format}{%
\subsection{Wide to Long Format}\label{wide-to-long-format}}

In wide format, each subject's responses are in a single row. In long
format, each row is a single subject-variable combination.

Let's create a wide format dataframe:

\begin{Shaded}
\begin{Highlighting}[]
\NormalTok{wide\_df }\OtherTok{\textless{}{-}} \FunctionTok{data.frame}\NormalTok{(}
  \AttributeTok{ID =} \DecValTok{1}\SpecialCharTok{:}\DecValTok{2}\NormalTok{,}
  \AttributeTok{Math =} \FunctionTok{c}\NormalTok{(}\DecValTok{35}\NormalTok{, }\DecValTok{32}\NormalTok{, }\DecValTok{48}\NormalTok{, }\DecValTok{44}\NormalTok{),}
  \AttributeTok{English =} \FunctionTok{c}\NormalTok{(}\DecValTok{92}\NormalTok{, }\DecValTok{88}\NormalTok{, }\DecValTok{95}\NormalTok{, }\DecValTok{89}\NormalTok{),}
  \AttributeTok{Science =} \FunctionTok{c}\NormalTok{(}\DecValTok{49}\NormalTok{, }\DecValTok{85}\NormalTok{, }\DecValTok{40}\NormalTok{,}\DecValTok{55}\NormalTok{),}
  \AttributeTok{Year =}\FunctionTok{c}\NormalTok{(}\DecValTok{2023}\NormalTok{, }\DecValTok{2023}\NormalTok{, }\DecValTok{2024}\NormalTok{,}\DecValTok{2024}\NormalTok{)}
\NormalTok{)}

\FunctionTok{print}\NormalTok{(wide\_df)}
\end{Highlighting}
\end{Shaded}

\begin{verbatim}
##   ID Math English Science Year
## 1  1   35      92      49 2023
## 2  2   32      88      85 2023
## 3  1   48      95      40 2024
## 4  2   44      89      55 2024
\end{verbatim}

\hypertarget{using-melt-from-reshape2}{%
\subsubsection{Using melt() from
reshape2}\label{using-melt-from-reshape2}}

\begin{Shaded}
\begin{Highlighting}[]
\NormalTok{long\_df\_melt }\OtherTok{\textless{}{-}} \FunctionTok{melt}\NormalTok{(wide\_df, }\AttributeTok{id.vars =} \StringTok{"ID"}\NormalTok{, }
                     \AttributeTok{variable.name =} \StringTok{"Subject"}\NormalTok{, }\AttributeTok{value.name =} \StringTok{"Score"}\NormalTok{)}

\FunctionTok{print}\NormalTok{(long\_df\_melt)}
\end{Highlighting}
\end{Shaded}

\begin{verbatim}
##    ID Subject Score
## 1   1    Math    35
## 2   2    Math    32
## 3   1    Math    48
## 4   2    Math    44
## 5   1 English    92
## 6   2 English    88
## 7   1 English    95
## 8   2 English    89
## 9   1 Science    49
## 10  2 Science    85
## 11  1 Science    40
## 12  2 Science    55
## 13  1    Year  2023
## 14  2    Year  2023
## 15  1    Year  2024
## 16  2    Year  2024
\end{verbatim}

Explanation:

\texttt{melt()} is similar to gather() but from an older package
\texttt{id.vars} specifies which columns to keep as is
\texttt{variable.name} and \texttt{value.name} specify names for the new
columns

\hypertarget{reshape-the-data-to-long-format-excluding-the-year-column}{%
\subsubsection{Reshape the data to long format, excluding the Year
column}\label{reshape-the-data-to-long-format-excluding-the-year-column}}

\begin{Shaded}
\begin{Highlighting}[]
\NormalTok{long\_df\_melt1 }\OtherTok{\textless{}{-}} \FunctionTok{pivot\_longer}\NormalTok{(wide\_df, }
                        \AttributeTok{cols =} \FunctionTok{c}\NormalTok{(Math, English, Science), }
                        \AttributeTok{names\_to =} \StringTok{"Subject"}\NormalTok{, }
                        \AttributeTok{values\_to =} \StringTok{"Score"}\NormalTok{)}

\FunctionTok{print}\NormalTok{(long\_df\_melt1)}
\end{Highlighting}
\end{Shaded}

\begin{verbatim}
## # A tibble: 12 x 4
##       ID  Year Subject Score
##    <int> <dbl> <chr>   <dbl>
##  1     1  2023 Math       35
##  2     1  2023 English    92
##  3     1  2023 Science    49
##  4     2  2023 Math       32
##  5     2  2023 English    88
##  6     2  2023 Science    85
##  7     1  2024 Math       48
##  8     1  2024 English    95
##  9     1  2024 Science    40
## 10     2  2024 Math       44
## 11     2  2024 English    89
## 12     2  2024 Science    55
\end{verbatim}

\hypertarget{using-gather-from-tidyr}{%
\subsubsection{Using gather() from
tidyr}\label{using-gather-from-tidyr}}

\begin{Shaded}
\begin{Highlighting}[]
\NormalTok{long\_df\_gather }\OtherTok{\textless{}{-}}\NormalTok{ wide\_df }\SpecialCharTok{\%\textgreater{}\%}
  \FunctionTok{gather}\NormalTok{(}\AttributeTok{key =} \StringTok{"Subject"}\NormalTok{, }\AttributeTok{value =} \StringTok{"Score"}\NormalTok{, }\SpecialCharTok{{-}}\FunctionTok{c}\NormalTok{(ID,Year))}

\FunctionTok{print}\NormalTok{(long\_df\_gather)}
\end{Highlighting}
\end{Shaded}

\begin{verbatim}
##    ID Year Subject Score
## 1   1 2023    Math    35
## 2   2 2023    Math    32
## 3   1 2024    Math    48
## 4   2 2024    Math    44
## 5   1 2023 English    92
## 6   2 2023 English    88
## 7   1 2024 English    95
## 8   2 2024 English    89
## 9   1 2023 Science    49
## 10  2 2023 Science    85
## 11  1 2024 Science    40
## 12  2 2024 Science    55
\end{verbatim}

Explanation:

\texttt{gather()} takes all columns except ID and creates two new
columns:

\texttt{Subject}: contains the original column names (Math, English,
Science)

\texttt{Score}: contains the values from those columns

\hypertarget{using-pivot_longer-from-tidyverse-preferred}{%
\subsection{Using pivot\_longer from tidyverse
(preferred)}\label{using-pivot_longer-from-tidyverse-preferred}}

For this course, we will be using \texttt{pivot\_longer} from the
\texttt{tidyverse} package frequently. The reason is that, like other
similar functions discussed above, \texttt{pivot\_longer} allows us to
transform data from a wide format to a long format, which is essential
for many types of data analysis and visualization tasks. More on in the
next section:

\hypertarget{preparing-data-for-visualization}{%
\section{Preparing Data for
Visualization}\label{preparing-data-for-visualization}}

\texttt{pivot\_longer}: This function is particularly useful because it
converts multiple columns into key-value pairs, creating a single column
for the variable names and another for the values. This is crucial when
we need a single column for the x or y axis in our plots. In wide format
data, having multiple columns for what we want to plot can complicate
the visualization process.

\texttt{summarize}: We use this function to calculate summary statistics
such as average percentages for each subject across all rows.

\texttt{starts\_with}: This function helps us select columns that start
with a specific string, which is useful when dealing with data that has
multiple similarly named columns, like percentages in this example.

\begin{Shaded}
\begin{Highlighting}[]
\CommentTok{\# Transform from wide to long using pivot\_longer}
\NormalTok{long\_df\_pivot }\OtherTok{\textless{}{-}} \FunctionTok{pivot\_longer}\NormalTok{(wide\_df,}
            \AttributeTok{cols =} \SpecialCharTok{{-}}\FunctionTok{c}\NormalTok{(ID, Year), }\CommentTok{\# Columns to pivot (excluding ID and Year)}
            \AttributeTok{names\_to =} \StringTok{"Subject"}\NormalTok{, }\CommentTok{\# New column for variable names}
            \AttributeTok{values\_to =} \StringTok{"Score"}\NormalTok{)}\SpecialCharTok{\%\textgreater{}\%}  \CommentTok{\# New column for values}
  \FunctionTok{mutate}\NormalTok{(}\AttributeTok{ID =} \FunctionTok{as.factor}\NormalTok{(ID))  }\CommentTok{\# Convert ID to factor}


\CommentTok{\# Print the resulting long data frame}
\FunctionTok{print}\NormalTok{(long\_df\_pivot)}
\end{Highlighting}
\end{Shaded}

\begin{verbatim}
## # A tibble: 12 x 4
##    ID     Year Subject Score
##    <fct> <dbl> <chr>   <dbl>
##  1 1      2023 Math       35
##  2 1      2023 English    92
##  3 1      2023 Science    49
##  4 2      2023 Math       32
##  5 2      2023 English    88
##  6 2      2023 Science    85
##  7 1      2024 Math       48
##  8 1      2024 English    95
##  9 1      2024 Science    40
## 10 2      2024 Math       44
## 11 2      2024 English    89
## 12 2      2024 Science    55
\end{verbatim}

\hypertarget{preparing-data-for-plotting}{%
\subsubsection{Preparing data for
plotting}\label{preparing-data-for-plotting}}

Long format and summarizing the data are important steps to plotting.
Let plots the average scores for each course using ggplot by following
the steps below:

\hypertarget{step-1-use-mutate-to-calculate-percentages-in-a-new-vector}{%
\subsection{Step 1: Use mutate to calculate percentages in a new
vector}\label{step-1-use-mutate-to-calculate-percentages-in-a-new-vector}}

\begin{Shaded}
\begin{Highlighting}[]
\NormalTok{data\_percent }\OtherTok{\textless{}{-}}\NormalTok{ wide\_df }\SpecialCharTok{\%\textgreater{}\%}
  \FunctionTok{mutate}\NormalTok{(}\AttributeTok{total =}\NormalTok{ Math }\SpecialCharTok{+}\NormalTok{ English }\SpecialCharTok{+}\NormalTok{ Science) }\SpecialCharTok{\%\textgreater{}\%}
  \FunctionTok{mutate}\NormalTok{(}\AttributeTok{percent\_Math =}\NormalTok{ Math }\SpecialCharTok{/}\NormalTok{ total }\SpecialCharTok{*} \DecValTok{100}\NormalTok{,}
         \AttributeTok{percent\_English =}\NormalTok{ English }\SpecialCharTok{/}\NormalTok{ total }\SpecialCharTok{*} \DecValTok{100}\NormalTok{,}
         \AttributeTok{percent\_Science =}\NormalTok{ Science }\SpecialCharTok{/}\NormalTok{ total }\SpecialCharTok{*} \DecValTok{100}\NormalTok{)}
\FunctionTok{print}\NormalTok{(data\_percent)}
\end{Highlighting}
\end{Shaded}

\begin{verbatim}
##   ID Math English Science Year total percent_Math percent_English
## 1  1   35      92      49 2023   176     19.88636        52.27273
## 2  2   32      88      85 2023   205     15.60976        42.92683
## 3  1   48      95      40 2024   183     26.22951        51.91257
## 4  2   44      89      55 2024   188     23.40426        47.34043
##   percent_Science
## 1        27.84091
## 2        41.46341
## 3        21.85792
## 4        29.25532
\end{verbatim}

\hypertarget{step-2-calculate-averagemean-percentages-per-course}{%
\subsection{Step 2: Calculate average/mean percentages per
course}\label{step-2-calculate-averagemean-percentages-per-course}}

\begin{Shaded}
\begin{Highlighting}[]
\DocumentationTok{\#\# Calculate average percentages}
\NormalTok{avg\_percent }\OtherTok{\textless{}{-}}\NormalTok{ data\_percent }\SpecialCharTok{\%\textgreater{}\%}
  \FunctionTok{summarise}\NormalTok{(}\AttributeTok{avg\_Math =} \FunctionTok{mean}\NormalTok{(percent\_Math),}
            \AttributeTok{avg\_English =} \FunctionTok{mean}\NormalTok{(percent\_English),}
            \AttributeTok{avg\_Science =} \FunctionTok{mean}\NormalTok{(percent\_Science))}

\NormalTok{avg\_percent}
\end{Highlighting}
\end{Shaded}

\begin{longtable}[]{@{}rrr@{}}
\toprule\noalign{}
avg\_Math & avg\_English & avg\_Science \\
\midrule\noalign{}
\endhead
\bottomrule\noalign{}
\endlastfoot
21.28247 & 48.61314 & 30.10439 \\
\end{longtable}

\hypertarget{step-3-use-starts_with-in-pivot_longer-to-pivot-data-to-long}{%
\subsection{Step 3: use starts\_with (in pivot\_longer) to pivot data to
long}\label{step-3-use-starts_with-in-pivot_longer-to-pivot-data-to-long}}

\begin{Shaded}
\begin{Highlighting}[]
\NormalTok{data\_long }\OtherTok{\textless{}{-}}\NormalTok{ avg\_percent }\SpecialCharTok{\%\textgreater{}\%}
  \FunctionTok{pivot\_longer}\NormalTok{(}\AttributeTok{cols =} \FunctionTok{starts\_with}\NormalTok{(}\StringTok{"avg\_"}\NormalTok{),}
               \AttributeTok{names\_to =} \StringTok{"Subject"}\NormalTok{,}
               \AttributeTok{values\_to =} \StringTok{"Percentage"}\NormalTok{) }
\FunctionTok{print}\NormalTok{(data\_long)}
\end{Highlighting}
\end{Shaded}

\begin{verbatim}
## # A tibble: 3 x 2
##   Subject     Percentage
##   <chr>            <dbl>
## 1 avg_Math          21.3
## 2 avg_English       48.6
## 3 avg_Science       30.1
\end{verbatim}

\hypertarget{step-4-rename-columns}{%
\subsection{Step 4: Rename columns}\label{step-4-rename-columns}}

\begin{Shaded}
\begin{Highlighting}[]
\NormalTok{avg\_percent }\OtherTok{\textless{}{-}}\NormalTok{ data\_long }\SpecialCharTok{\%\textgreater{}\%}
  \FunctionTok{mutate}\NormalTok{(}\AttributeTok{Subject =} \FunctionTok{factor}\NormalTok{(Subject, }\AttributeTok{levels =} \FunctionTok{c}\NormalTok{(}\StringTok{"avg\_Math"}\NormalTok{, }\StringTok{"avg\_English"}\NormalTok{, }\StringTok{"avg\_Science"}\NormalTok{),}
                          \AttributeTok{labels =} \FunctionTok{c}\NormalTok{(}\StringTok{"Math"}\NormalTok{, }\StringTok{"English"}\NormalTok{, }\StringTok{"Science"}\NormalTok{)))}

\FunctionTok{print}\NormalTok{(avg\_percent)}
\end{Highlighting}
\end{Shaded}

\begin{verbatim}
## # A tibble: 3 x 2
##   Subject Percentage
##   <fct>        <dbl>
## 1 Math          21.3
## 2 English       48.6
## 3 Science       30.1
\end{verbatim}

\hypertarget{step-5-plot-the-data}{%
\subsection{Step 5: Plot the data}\label{step-5-plot-the-data}}

\begin{Shaded}
\begin{Highlighting}[]
\CommentTok{\# Plotting using ggplot2}
\FunctionTok{ggplot}\NormalTok{(avg\_percent, }\FunctionTok{aes}\NormalTok{(}\AttributeTok{x =}\NormalTok{ Subject, }\AttributeTok{y =}\NormalTok{ Percentage, }\AttributeTok{fill =}\NormalTok{ Subject)) }\SpecialCharTok{+}
  \FunctionTok{geom\_bar}\NormalTok{(}\AttributeTok{stat =} \StringTok{"identity"}\NormalTok{) }\SpecialCharTok{+}
  \FunctionTok{labs}\NormalTok{(}\AttributeTok{title =} \StringTok{"Percentage Distribution by Subject"}\NormalTok{,}
       \AttributeTok{x =} \StringTok{"Subject"}\NormalTok{, }\AttributeTok{y =} \StringTok{"Percentage"}\NormalTok{) }\SpecialCharTok{+}
  \FunctionTok{theme\_minimal}\NormalTok{()}
\end{Highlighting}
\end{Shaded}

\includegraphics{Merge_Join_USDI_files/figure-latex/unnamed-chunk-21-1.pdf}

\hypertarget{step-6-plotting-with-customizations}{%
\subsection{Step 6: Plotting with
customizations}\label{step-6-plotting-with-customizations}}

\begin{Shaded}
\begin{Highlighting}[]
\CommentTok{\# Define custom color palette}
\NormalTok{custom\_colors }\OtherTok{\textless{}{-}} \FunctionTok{c}\NormalTok{(}\StringTok{"\#126e96"}\NormalTok{, }\StringTok{"\#8B4513"}\NormalTok{, }\StringTok{"\#CD853F"}\NormalTok{) }

\CommentTok{\# Plotting with customizations}
\NormalTok{plot }\OtherTok{\textless{}{-}} \FunctionTok{ggplot}\NormalTok{(avg\_percent, }\FunctionTok{aes}\NormalTok{(}\AttributeTok{x =}\NormalTok{ Subject, }\AttributeTok{y =}\NormalTok{ Percentage, }\AttributeTok{fill =}\NormalTok{ Subject)) }\SpecialCharTok{+}
  \FunctionTok{geom\_bar}\NormalTok{(}\AttributeTok{stat =} \StringTok{"identity"}\NormalTok{) }\SpecialCharTok{+}
  \FunctionTok{scale\_fill\_manual}\NormalTok{(}\AttributeTok{values =}\NormalTok{ custom\_colors) }\SpecialCharTok{+}
  \FunctionTok{labs}\NormalTok{(}\AttributeTok{title =} \StringTok{"Percentage Distribution by Subject"}\NormalTok{,}
       \AttributeTok{x =} \StringTok{"Subject"}\NormalTok{, }\AttributeTok{y =} \StringTok{"Percentage"}\NormalTok{) }\SpecialCharTok{+}
  \FunctionTok{theme\_minimal}\NormalTok{() }\SpecialCharTok{+}
  \FunctionTok{theme}\NormalTok{(}\AttributeTok{plot.title =} \FunctionTok{element\_text}\NormalTok{(}\AttributeTok{hjust =} \FloatTok{0.5}\NormalTok{), }\AttributeTok{legend.position =} \StringTok{"bottom"}\NormalTok{) }\SpecialCharTok{+}
  \FunctionTok{ylab}\NormalTok{(}\StringTok{"Percentage"}\NormalTok{) }\SpecialCharTok{+}
  \FunctionTok{geom\_text}\NormalTok{(}\FunctionTok{aes}\NormalTok{(}\AttributeTok{label =} \FunctionTok{paste0}\NormalTok{(}\FunctionTok{round}\NormalTok{(Percentage), }\StringTok{"\%"}\NormalTok{)), }\AttributeTok{hjust =} \FloatTok{3.5}\NormalTok{)}\SpecialCharTok{+}
  \FunctionTok{coord\_flip}\NormalTok{()}

\CommentTok{\# Print the plot}
\FunctionTok{print}\NormalTok{(plot)}
\end{Highlighting}
\end{Shaded}

\includegraphics{Merge_Join_USDI_files/figure-latex/unnamed-chunk-22-1.pdf}

\hypertarget{step-7-plotting-with-customizations-with-geom_point-and-faceting}{%
\subsection{Step 7: Plotting with customizations with geom\_point and
faceting}\label{step-7-plotting-with-customizations-with-geom_point-and-faceting}}

\hypertarget{scores-by-year-for-each-subject-per-student}{%
\subsubsection{Scores by Year for Each Subject per
Student}\label{scores-by-year-for-each-subject-per-student}}

\begin{Shaded}
\begin{Highlighting}[]
\CommentTok{\# Ensure the Year are treated as factors}
\NormalTok{long\_df\_pivot}\SpecialCharTok{$}\NormalTok{Year}\OtherTok{\textless{}{-}} \FunctionTok{as.factor}\NormalTok{(long\_df\_pivot}\SpecialCharTok{$}\NormalTok{Year)}

\CommentTok{\# Plotting scores by Year for Each Subject with points, lines, and y{-}axis labels}
\NormalTok{plot\_points }\OtherTok{\textless{}{-}} \FunctionTok{ggplot}\NormalTok{(long\_df\_pivot, }\FunctionTok{aes}\NormalTok{(}\AttributeTok{x =}\NormalTok{ Subject, }\AttributeTok{y =}\NormalTok{ Score, }\AttributeTok{color =}\NormalTok{ Subject)) }\SpecialCharTok{+}
  \FunctionTok{geom\_point}\NormalTok{(}\AttributeTok{linewidth =} \DecValTok{3}\NormalTok{) }\SpecialCharTok{+}
  \FunctionTok{geom\_line}\NormalTok{(}\FunctionTok{aes}\NormalTok{(}\AttributeTok{group =}\NormalTok{ ID), }\AttributeTok{linewidth =} \DecValTok{1}\NormalTok{) }\SpecialCharTok{+}  \CommentTok{\# Add a line connecting points for each ID}
  \FunctionTok{geom\_text}\NormalTok{(}\FunctionTok{aes}\NormalTok{(}\AttributeTok{label =}\NormalTok{ Score), }\AttributeTok{vjust=}\FloatTok{0.9}\NormalTok{, }\AttributeTok{hjust =} \SpecialCharTok{{-}}\FloatTok{1.0}\NormalTok{, }\AttributeTok{size =} \DecValTok{3}\NormalTok{) }\SpecialCharTok{+}  \CommentTok{\# Add y{-}axis labels}
  \FunctionTok{scale\_color\_manual}\NormalTok{(}\AttributeTok{values =}\NormalTok{ custom\_colors) }\SpecialCharTok{+}
  \FunctionTok{labs}\NormalTok{(}\AttributeTok{title =} \StringTok{"Scores by Year for Each Subject per Student"}\NormalTok{,}
       \AttributeTok{x =} \StringTok{"Subject"}\NormalTok{, }\AttributeTok{y =} \StringTok{"Score"}\NormalTok{) }\SpecialCharTok{+}
  \FunctionTok{theme\_minimal}\NormalTok{() }\SpecialCharTok{+}
  \FunctionTok{theme}\NormalTok{(}\AttributeTok{plot.title =} \FunctionTok{element\_text}\NormalTok{(}\AttributeTok{vjust =} \FloatTok{1.5}\NormalTok{), }\AttributeTok{legend.position =} \StringTok{"bottom"}\NormalTok{) }\SpecialCharTok{+}
  \FunctionTok{facet\_grid}\NormalTok{(Year }\SpecialCharTok{\textasciitilde{}}\NormalTok{ ID, }\AttributeTok{scales =} \StringTok{"free\_x"}\NormalTok{)  }\CommentTok{\# Facet by Year}
\end{Highlighting}
\end{Shaded}

\begin{verbatim}
## Warning in geom_point(linewidth = 3): Ignoring unknown parameters: `linewidth`
\end{verbatim}

\begin{Shaded}
\begin{Highlighting}[]
\CommentTok{\# Print the plot}
\FunctionTok{print}\NormalTok{(plot\_points)}
\end{Highlighting}
\end{Shaded}

\includegraphics{Merge_Join_USDI_files/figure-latex/unnamed-chunk-23-1.pdf}

\hypertarget{creating-plot-for-the-entire-database}{%
\section{Creating plot for the entire
database}\label{creating-plot-for-the-entire-database}}

\hypertarget{using-the-wide-database}{%
\subsection{Using the wide database}\label{using-the-wide-database}}

Creating plots using the wide format data can be beneficial when you
want to plot multiple data points rather than summary statistics (mean,
mode etc).

In a wide format, each column typically represents a different variable
or measurement, making it easier to plot relationships between these
variables directly.

This approach is often more straightforward for scatter plots and other
visualizations that involve pairwise comparisons or multiple variables.

\hypertarget{using-the-facet_grid}{%
\subsection{Using the facet\_grid}\label{using-the-facet_grid}}

\hypertarget{explanation}{%
\section{Explanation:}\label{explanation}}

Using the long format data can also allow for more flexibility in
visualizing data by group. The long format typically involves having one
column for the variable names and another for the values, which makes it
easier to apply faceting.

Faceting, such as with \texttt{facet\_grid} or \texttt{facet\_wrap},
allows you to create multiple plots based on the levels of one or more
grouping variables. This is particularly useful for comparing subgroups
within your data, as each subgroup gets its own individual plot.

\begin{Shaded}
\begin{Highlighting}[]
\CommentTok{\# Ensure the IDs are treated as factors}
\NormalTok{wide\_df}\SpecialCharTok{$}\NormalTok{ID }\OtherTok{\textless{}{-}} \FunctionTok{as.factor}\NormalTok{(wide\_df}\SpecialCharTok{$}\NormalTok{ID)}

\CommentTok{\# Plotting scores by student with geom\_point and geom\_text}
\NormalTok{plot\_points1 }\OtherTok{\textless{}{-}} \FunctionTok{ggplot}\NormalTok{(wide\_df, }\FunctionTok{aes}\NormalTok{(}\AttributeTok{x =}\NormalTok{ Math, }\AttributeTok{y =}\NormalTok{ English, }\AttributeTok{color =}\NormalTok{ ID)) }\SpecialCharTok{+}
  \FunctionTok{geom\_point}\NormalTok{(}\AttributeTok{size =} \DecValTok{3}\NormalTok{) }\SpecialCharTok{+}
  \FunctionTok{geom\_text}\NormalTok{(}\FunctionTok{aes}\NormalTok{(}\AttributeTok{label =} \FunctionTok{paste0}\NormalTok{(}\StringTok{"("}\NormalTok{, Math, }\StringTok{", "}\NormalTok{, English, }\StringTok{")"}\NormalTok{)), }\AttributeTok{vjust =} \FloatTok{1.5}\NormalTok{, }\AttributeTok{size =} \DecValTok{3}\NormalTok{) }\SpecialCharTok{+}
  \FunctionTok{scale\_color\_manual}\NormalTok{(}\AttributeTok{values =}\NormalTok{ custom\_colors) }\SpecialCharTok{+}
  \FunctionTok{labs}\NormalTok{(}\AttributeTok{title =} \StringTok{"Math vs. English Scores by Student"}\NormalTok{,}
       \AttributeTok{x =} \StringTok{"Math Score"}\NormalTok{, }\AttributeTok{y =} \StringTok{"English Score"}\NormalTok{) }\SpecialCharTok{+}
  \FunctionTok{theme\_minimal}\NormalTok{() }\SpecialCharTok{+}
  \FunctionTok{theme}\NormalTok{(}\AttributeTok{plot.title =} \FunctionTok{element\_text}\NormalTok{(}\AttributeTok{hjust =} \FloatTok{0.5}\NormalTok{), }\AttributeTok{legend.position =} \StringTok{"top"}\NormalTok{)}

\CommentTok{\# Print the plot}
\FunctionTok{print}\NormalTok{(plot\_points1)}
\end{Highlighting}
\end{Shaded}

\includegraphics{Merge_Join_USDI_files/figure-latex/unnamed-chunk-24-1.pdf}

\hypertarget{long-to-wide-format}{%
\section{Long to Wide Format}\label{long-to-wide-format}}

Now let's convert our long format data back to wide format.

\hypertarget{using-spread-from-tidyr}{%
\subsection{Using spread() from tidyr}\label{using-spread-from-tidyr}}

\begin{verbatim}
##   ID Year English Math Science
## 1  1 2023      92   35      49
## 2  1 2024      95   48      40
## 3  2 2023      88   32      85
## 4  2 2024      89   44      55
\end{verbatim}

Explanation:

\texttt{spread()} is the opposite of \texttt{gather()} It takes the
\texttt{Subject} column and spreads it out into separate columns The
values in these new columns come from the \texttt{Score} column

\end{document}
